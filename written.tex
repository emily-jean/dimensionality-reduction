\documentclass{neu_handout}
\usepackage{url}
\usepackage{amssymb}
\usepackage{amsmath}
\usepackage{marvosym}
\usepackage{graphicx}
\graphicspath{ {images/} }
\everymath{\displaystyle}

% Professor/Course information
\title{HW3 Part 1}
\author{Emily Dutile}
\date{November 2017}
\course{CS6220}{Data Mining Techniques}

\begin{document}

\section*{1. PCA Eigenvector Orthogonality}

$$A\vec{x} = \lambda_1 \vec{x}$$
$$A\vec{y} = \lambda_2 \vec{y}$$

where A is a symmetric matrix, ${\vec{x}}$ and $\vec{y}$ are the eignvectors that correspond, respectively, to eigenvalues
$\lambda1$ and $\lambda2$. Show mathematically that $\vec{x}$ and $\vec{y}$ must be orthogonal if the eigenvalues are different

$$ \lambda_1 \vec{x^T} = A \vec{x^T} $$
$$ \lambda_1 \vec{x^T} \vec{y} = \vec{x^T} A \vec{y} = \vec{x^T} \lambda \vec{y} $$
$$ (\lambda_1 - \lambda_2) \vec{x^T} \vec{y} = 0 $$
$$ \vec{x^T} \vec{y} = 0 $$

\end{document}
